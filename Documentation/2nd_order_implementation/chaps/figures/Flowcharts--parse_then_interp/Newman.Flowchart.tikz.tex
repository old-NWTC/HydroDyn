%%% This approach is not correct: we don't want to parse first.  We can do the parsing as part of the loop and interpolate within the loop.
\begin{tikzpicture}

% temporary node for the sectioning labels
\node[tempnode]
      (TmpNode) {};

% WAMIT 1st order
\node[info,right of=TmpNode,node distance=18mm,text width=19mm,text centered] 
      (W1Data) {Read \nth{1}~order WAMIT~output (.7~.8~.9)};

\node[process,below of=W1Data,node distance=13mm]
      (W1Sort) {Sort data};
   \draw[cl] (W1Data) -- (W1Sort);

\node[process,below of=W1Sort,node distance=12mm]
      (W1Convert) {Convert $\tau \rightarrow \omega$};
   \draw[cl] (W1Sort) -- (W1Convert);

\node[info,below of=W1Convert,node distance=13mm]
      (W1Info) {Raw \nth{1} order data};
   \draw[cl] (W1Convert) -- (W1Info);

% unidirectional
   \node[process]
         (W1UniParse) at ($(W1Info)+(-0mm,-13mm)$) {$\beta_1=\beta_2$ terms};
      \draw[cl] (W1Info) -- (W1UniParse);
   
   

%WAMIT 2nd order
\node[info,right of=W1Data,node distance=30mm,text width=21mm,text centered] 
      (W2Data) {Read \nth{2}~order WAMIT~output (.10d~.11d~.12d)};

\node[process,below of=W2Data,node distance=13mm]
      (W2Sort) {Sort data};
   \draw[cl] (W2Data) -- (W2Sort);

\node[process,below of=W2Sort,node distance=12mm]
      (W2Convert) {Convert $\tau \rightarrow \omega$};
   \draw[cl] (W2Sort) -- (W2Convert);

\node[info,below of=W2Convert,node distance=13mm]
      (W2Info) {Raw \nth{2} order data};
   \draw[cl] (W2Convert) -- (W2Info);


% unidirectional
   \node[process,text width=24mm,text centered]
         (W2UniParse) at ($(W2Info)+(-0mm,-13mm)$) {$\beta_1=\beta_2$ when $\omega_1=\omega_2$ terms};
      \draw[cl] (W2Info) -- (W2UniParse);
   


% Interpolation step

% unidirectional
\node[process,anchor=north]
      (UniInterp1) at ($(W1Data)+(15mm,-65mm)$) {Interpolate $\beta$};
   \draw[cl] (W1UniParse) -- (UniInterp1);
   \draw[cl] (W2UniParse) -- (UniInterp1);

\node[process,below of=UniInterp1]
      (UniInterp2) {Interpolate $\omega$};
   \draw[cl] (UniInterp1) -- (UniInterp2);

%set the background box of unidirectional
\begin{pgfonlayer}{background}
\filldraw[rounded corners=3mm,draw=blue!30!black!35,
            fill=white!80!blue!20!red!20!black!10]
      ($(UniInterp1.north west) +(-12mm,2mm)$) 
         rectangle ($(UniInterp2.south east)+(2mm,-2mm)$);
\end{pgfonlayer}
\node[tempnode,anchor=north,font=\footnotesize,text centered,text width=18mm,rotate=90]
      (UniDir) at ($(UniInterp1.west)!.5!(UniInterp2.west)+(-11mm,0mm)$) {\it{Uni-dir waves}\\(2$\times$1D)};


%Transfer function
\node[info,below of=UniInterp2,node distance=13mm]
      (TransFunc) {$F^{-}_k(\omega_m,\omega_m)$};
   \draw[cl] (UniInterp2) -- (TransFunc);

%EQ
\node[process,below of=TransFunc]
      (EQ) {\Cref{eq:Newman}};
   \draw[cl] (TransFunc) -- (EQ);



%output
\node[info,below of=EQ,text width=21mm,text centered,node distance=14mm]
      (Output) {Forces and moments to apply to point mesh};
   \draw[cl] (EQ) -- (Output);




%% draw some lines along the side for marking what parts do what
\coordinate (tmp1) at ($(TmpNode |- W1Data)+(0mm,7mm)$);
\coordinate (tmp2) at ($(TmpNode |- W1Info)+(0mm,-5mm)$);
\draw[black]
      ($(tmp1)+(3mm,0mm)$) -- (tmp1) -- (tmp2) 
         node[midway,above,sloped,rotate=180,font=\small] {Data reading} -- +(3mm,0mm);

\coordinate (tmp1) at ($(TmpNode |- W1UniParse)+(0mm,5mm)$);
\coordinate (tmp2) at ($(TmpNode |- W1UniParse)+(0mm,-5mm)$);
\draw[black]
      ($(tmp1)+(3mm,0mm)$) -- (tmp1) -- (tmp2) 
            node[midway,above,sloped,rotate=180,font=\small] {Parsing}-- +(3mm,0mm);

\coordinate (tmp1) at ($(TmpNode |- UniInterp1)+(0mm,5mm)$);
\coordinate (tmp2) at ($(TmpNode |- UniInterp2)+(0mm,-5mm)$);
\draw[black,text width=21mm,text centered]
      ($(tmp1)+(3mm,0mm)$) -- (tmp1) -- (tmp2) 
            node[midway,above,sloped,rotate=180,font=\small] {Interpolation (\Cref{chap:Algorithm})}-- +(3mm,0mm);

   
\end{tikzpicture}
\endinput




