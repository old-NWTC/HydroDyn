\chapter{Integration of the WAMIT2 module within HydroDyn}
\label{chap:WAMIT2:Integration}

In order to integrate the {\tt WAMIT2} module into \HD, several changes need to be made to the \HD input file and to the file reading and parsing routines within the \fname{HydroDyn\_Input.f90} file.  Additional changes to the input file are required for multi-directional waves (see \Cref{chap:MultiDir,chap:Waves2:Integration}).

\begin{center}
   \begin{minipage}[t]{\linewidth}
      \fvset{frame=single,fontsize=\scriptsize,numbers=left,numbersep=3pt,obeytabs,tabsize=1,fontfamily=fvm,commentchar=\%}
      \VerbatimInput{chaps/HD_Input/HD_Input_File_WAMIT2.txt}
   \end{minipage}
   \captionof{table}{New section for the \HD input file for the second order forces calculated by the WAMIT2 module.}
   \label{tab:HD_WAMIT2_InputMods}
\end{center}

The additional second order waves information that needs to be added into the \HD input file is give in \Cref{tab:HD_WAMIT2_InputMods}.  This is inserted between the sections marked {\tt FLOATING PLATFORM} and {\tt FLOATING PLATFORM FORCE FLAGS}.  To decide which force components are calculated, the \rname{HydroDyn\_Input} copies the component direction information from the {\tt FLOATING PLATFORM FORCE FLAGS} section to the \rname{WAMIT2\%InitInput}.  The \rname{WAMIT2\_Init} subroutine decides which of the force components can be calculated based on the information available within the WAMIT output file and method chosen.

Add in section here describing what combinations of MnDrift, NewmanApp, DiffQTF, and SumQTF are allowed -- table perhaps?

Add in description of the inputs with more detail about when used etc.


